
%this tex file was auto produced from TEI by lbp-print-xslt 1.0.0 critical stylesheets on 2022-08-10T19:39:07.443717Z using the  file:/usr/src/app/annotations.xslt 
\documentclass[twoside, openright]{report}




%imakeidx must be loaded beore eledmac
\usepackage{imakeidx}

\usepackage{reledmac}
% adds linked internal references
\usepackage[hidelinks]{hyperref}
\usepackage{titlesec}

\usepackage [latin]{babel}
\usepackage[style=american]{csquotes}
\usepackage{geometry}
\usepackage{fancyhdr}
\usepackage[letter, center, cam]{crop}


\geometry{paperheight=10in, paperwidth=7in, hmarginratio=3:2, inner=1.7in, outer=1.13in, bmargin=1in} 

%fancyheading settings
\pagestyle{fancy}

%git package 
\usepackage{gitinfo2}

%watermark




%quotes settings
\MakeOuterQuote{"}

%package for angle brackets
\usepackage{textcomp}

%title settings
\titleformat{\section}{\normalfont\scshape}{\thesection}{1em}{}
\titlespacing\section{0pt}{12pt plus 4pt minus 2pt}{12pt plus 2pt minus 2pt}
\titleformat{\chapter}{\normalfont\Large\uppercase}{\thechapter}{50pt}{}

%reledmac settings % these settings change footnotes to run inline as a paragraph, 
%change paragraph to twocol, threecol, or normal for different effects
\Xarrangement[A]{paragraph}
\Xarrangement[B]{paragraph}
\Xnotenumfont[A]{\normalfont\bfseries}
\Xnotenumfont[B]{\normalfont\bfseries}

\linenummargin{outer}
\sidenotemargin{inner}

%other settings
\linespread{1.1}

%custom macros
\newcommand{\name}[1]{\textsc{#1}}
\newcommand{\worktitle}[1]{\textit{#1}}




\newcommand{\crossref}[2]
{
\ifnum#1=000
#2
\else
Vide #1
\fi
}

\makeindex[name=sources,title=Index of sources,columns=1]


\begin{document}





\bigskip
http://scta.info/resoure/pgb1q1-cadanl/critical/transcription
\bigskip
\addcontentsline{toc}{chapter}{http://scta.info/resoure/pgb1q1-cadanl/critical/transcription}

\bigskip
Editor's note: paragraph 1 of pg-b1q1 added to print list
\bigskip

\beginnumbering

\pstart
\ledsidenote{\textbf{1}}
\edlabel{http://scta.info/resource/pgb1q1-cadanl}
\edtext{\enquote{Cupientes aliquid de penuria}}{\lemma{}\Afootnote[nosep]{
\crossref{\xlineref{http://scta.info/resource/pll1prol-cadzdd}}{\name{Lombardus}\index[persons]{}, \worktitle{Sent.}\index[works]{} I, prol. (I, 3, ll. 1).}}}\index[sources]{Peter Lombard!Sententiae!}, etc. Istud est prooemium libri \worktitle{Sententiarum}\index[works]{} qui liber dividitur in prooemium et tractatum. Secunda incipit ibi, \edtext{\enquote{veteris ac novae legis}}{\lemma{}\Afootnote[nosep]{
\crossref{\xlineref{http://scta.info/resource/pll1d1c1-vanspv}}{\name{Lombardus}\index[persons]{}, \worktitle{Sent.}\index[works]{} I, d. 1, c. 1 (I, 55, ll. 5).}}}\index[sources]{Peter Lombard!Sententiae!Distinctio 1}. 
\pend

\endnumbering


\bigskip
http://scta.info/resoure/lectio1/critical/transcription
\bigskip
\addcontentsline{toc}{chapter}{http://scta.info/resoure/lectio1/critical/transcription}

\bigskip
Editor's note: lectio 1 test
\bigskip

\beginnumbering
\section*{Lectio 1, de Fide} 
\bigskip
\section*{\textlangle{}Quaestio\textrangle{}} \section*{\textlangle{} Utrum in causa iudiciali fidei contra traditionem pure humanitus adinventam iudex idoneus ferret pro fide sententiam \textrangle{}} 
\pstart
\ledsidenote{\textbf{1}}
\edlabel{http://scta.info/resource/l1-cpspfs}
Circa prologum \worktitle{Sententiarum}\index[works]{}\edtext{}{\lemma{Sententiarum}\Bfootnote{Magistri Petri Plaoul \textit{in textu} R SV}} in quo \name{Magister}\index[persons]{} dicit  quod intentionis suae est \edtext{\enquote{munire Davidicam turrim vel potius munitam ostendere \edtext{clypeis}{\lemma{clypeis}\Bfootnote{\textit{om.} V}}}}{\lemma{}\Afootnote[nosep]{
\crossref{\xlineref{http://scta.info/resource/pll1prol-qifacc@11-18}}{ \name{Lombard}\index[persons]{}, \worktitle{Sentences}\index[works]{}, I, prologus., (I:1)}}}\index[sources]{!!} etc., quaero istam quaestionem: utrum \edtext{in}{\lemma{in}\Bfootnote{\textit{om.} R SV S}} causa iudiciali fidei \edtext{contra traditionem}{\lemma{contra traditionem}\Bfootnote{contradictionem R SV}} pure humanitus adinventam iudex idoneus ferret pro fide sententiam. 
\pend

\bigskip
\section*{\textlangle{}Conclusio\textrangle{}} 
\pstart
\ledsidenote{\textbf{2}}
\edlabel{http://scta.info/resource/l1-eipdlt}
Et inprimis protestatur quod fides non subicitur humano iudicio, et haec est una conclusio. Patet quia fides est donum Dei supernaturale et est de illis de quibus  \name{Iacobum Apostolus}\index[persons]{} dicit  \edtext{quod}{\lemma{quod}\Bfootnote{\textit{om.} R V}} \edtext{\enquote{omne datum optimum  
\ledsidenote{V5-va}
et omne donum perfectum desursum est descendens a Patre luminum}}{\lemma{}\Afootnote[nosep]{
\crossref{\xlineref{http://scta.info/resource/iac1_17}}{Iacobus 1:17}}}\index[sources]{x!Bible!Iacobi}. Et fundabitur haec conclusio \edtext{infra}{\lemma{infra}\Bfootnote{prima V}} per diversa media. Unde nisi haec conclusio esset vera, \edtext{sequeretur}{\lemma{sequeretur}\Bfootnote{sequitur S}} quod quis posset credere articulis fidei sine fide, hoc autem est falsum. Item  \worktitle{\textlangle{}secundae\textrangle{} Petri}\index[works]{}  dicitur quod \edtext{\enquote{Spiritu Sancto repleti locuti sunt sancti Dei homines}}{\lemma{}\Afootnote[nosep]{
\crossref{\xlineref{http://scta.info/resource/IIpetr1_21}}{II Petrus 1:21}}}\index[sources]{x!Bible!Petri II} et  \name{Hieronymus}\index[persons]{}  quod \edtext{\enquote{lex spiritualis est ideo revelatione indiget.}}{\lemma{}\Afootnote[nosep]{\name{S. Hieronymi}\index[persons]{}, \worktitle{Epistola LIII: Ad Paulinum}\index[works]{} (PL XXII:543)}}\index[sources]{!!} Et in  \worktitle{Psalmo}\index[works]{}  \edtext{\enquote{revela oculos et \edtext{considerabo}{\lemma{considerabo}\Bfootnote{considera R SV  considerata S}} mirabilia de lege tua.}}{\lemma{}\Afootnote[nosep]{
\crossref{\xlineref{http://scta.info/resource/ps118_18}}{Psalm 118:18}}}\index[sources]{x!Bible!Psalmi} 
\pend

\bigskip
\section*{\textlangle{}Difficultates circa fidem\textrangle{}} 
\pstart
\ledsidenote{\textbf{3}}
\edlabel{http://scta.info/resource/l1-shoatd}
Sed hic occurrunt arduae difficultates; et primo consideranda est descriptio fidei quam ponit  \name{Apostolus}\index[persons]{}, scilicet, \edtext{\enquote{fides \edtext{est}{\lemma{est}\Bfootnote{\textit{om.} R}} substantia rerum sperandarum, argumentum non apparentium.}}{\lemma{}\Afootnote[nosep]{
\crossref{\xlineref{http://scta.info/resource/hebr11_1}}{Hebrews 11:1}}}\index[sources]{x!Bible!ad Hebraeos} Ubi secundum \edtext{\name{Altissiodorensis}\index[persons]{} \edtext{in}{\lemma{in}\Bfootnote{\textit{om.} R SV S}} principio suae \worktitle{Summae}\index[works]{}}{\lemma{}\Afootnote[nosep]{Guillelmus Auxerre \worktitle{Summa aurea}\index[works]{}}}\index[sources]{!!} et \edtext{\name{\edtext{Guillelmum}{\lemma{Guillelmum}\Bfootnote{Guillelmi R}} Parisiensis}\index[persons]{} tractatu suo \worktitle{De fide et legibus}\index[works]{}}{\lemma{}\Afootnote[nosep]{Guillelmus Parisiensis, \worktitle{de fide et legibus}\index[works]{}}}\index[sources]{!!} sit una comparatio fidei, respectu credendorum, et caritatis, respectu amandorum; unde imaginatur quod sicut caritas dirigit hominem ad diligendum Deum propter se, ita proportionaliter fides inclinat intellectum ad credendum primae veritati propter se et \edtext{super}{\lemma{super}\Bfootnote{supra V}} omnia sine alia apparentia. Ideo fides est argumentum, et non est consequens nec conclusio. Ideo, sicut inquit  \name{\edtext{Guillelmus}{\lemma{Guillelmus}\Bfootnote{\textit{om.} V}} Altissiodorensis}\index[persons]{}  \edtext{\enquote{a quodam bene dictum est quod apud \name{Aristotelem}\index[persons]{} argumentum est ratio rei dubiae faciens fidem, apud autem Christum est\edtext{}{\lemma{est}\Bfootnote{\textit{iterum} R SV}} fides faciens rationem}}{\lemma{}\Afootnote[nosep]{Guillelmus Auxerre, \worktitle{Summa aurea}\index[works]{}, prologus.}}\index[sources]{!!}. Et hoc videtur esse contra \edtext{\name{Aureolem}\index[persons]{} prima quaestione Prologi \edtext{articulo}{\lemma{articulo}\Bfootnote{capitulo V}} primo}{\lemma{}\Afootnote[nosep]{Aureoli, Prologue, article 1}}\index[sources]{!!}, qui tenet quod articuli fidei sunt conclusiones ex aliis deductae, ad quas \edtext{processus 
\ledsidenote{Sb}
theologicus et}{\lemma{processus 
\ledsidenote{Sb}
theologicus et}\Bfootnote{\textit{om.} V}} processus theologici principaliter nituntur concludendas; et non sunt tamquam principia ex quibus alia theologice \edtext{deducuntur}{\lemma{deducuntur}\Bfootnote{deducantur V}}. 
\pend

\pstart
\ledsidenote{\textbf{4}}
\edlabel{http://scta.info/resource/l1-usaafe}
\edtext{Unde secundum  \name{Apostolum}\index[persons]{}  \edtext{\enquote{sine fide impossibile est Deo placere.}}{\lemma{}\Afootnote[nosep]{
\crossref{\xlineref{http://scta.info/resource/hebr11_6}}{\worktitle{Ad Hebraeos}\index[works]{} 11:6}}}\index[sources]{x!Bible!ad Hebraeos} Ideo in diffinitione bene dicitur: \edtext{\enquote{substantia sperandarum \edtext{sperandarum}{\lemma{sperandarum}\Bfootnote{separandarum S}} rerum, argumentum non apparentium. \edtext{apparentium}{\lemma{apparentium}\Bfootnote{apparentiam S}}}}{\lemma{}\Afootnote[nosep]{
\crossref{\xlineref{http://scta.info/resource/hebr11_1}}{\worktitle{Ad Hebraeos}\index[works]{} 11:1}}}\index[sources]{x!Bible!ad Hebraeos} Ideo fides dicitur substantia quia est totius 
\ledsidenote{Rb}
meriti fundamentum, nam sicut substantia dicitur subiectum accidentium, ita fides dicitur fundamentum omnium 
\ledsidenote{SVb}
aliarum veritatum. Ideo dicit  \name{Apostolus}\index[persons]{}: \edtext{\enquote{sine fide impossibile est placere Deo}}{\lemma{}\Afootnote[nosep]{
\crossref{\xlineref{http://scta.info/resource/hebr11_6}}{\worktitle{Ad Hebraeos}\index[works]{} 11:6}}}\index[sources]{x!Bible!ad Hebraeos}. Ideo dicitur \enquote*{substantia rerum}, non quod sit realiter substantia, immo est quod accidens animae inhaerens, sed sicut substantia dicitur perfectior et nobilior accidente, ita fides dicitur et est nobilior habitus animae infusus, ut tamquam fundamentum, cui aliae virtutes theologicae et cardinales innituntur ut fundamento, quia \edtext{regulat et dirigit}{\lemma{regulat et dirigit}\Bfootnote{regulat et dirigit \textit{corr. ex} dirigit et regulat R}} intellectum humanum respectu credendorum ad salutem \edtext{necessariorum}{\lemma{necessariorum}\Bfootnote{concipiendum et intelligendum \textit{add. sed del.} SV}}; nisi enim quis crediderit salvus esse non poterit. Elevat etiam humanum iudicium ad veritates theologicas concipiendum et intelligendum; nisi enim credideritis, non intelligetis, sicut dicit Christus. Et imaginandum est quod, sicut caritas dirigit hominem ad diligendum Deum propter se finaliter, ita fides similiter inclinat intellectum nostrum ad credendum articulis immediate propter Deum. Dicitur etiam fides \enquote*{non apparentium} intelligendum quantum est ex parte sui, quia \edtext{non}{\lemma{non}\Bfootnote{nonus SV}} \edtext{oportet}{\lemma{oportet}\Bfootnote{\textit{om.} R SV}} quod obiectum fidei sit de \edtext{per}{\lemma{per}\Bfootnote{\textit{om.} S}} se apparens et evidenter cognitum. Fides enim respectu sui obiecti est quaedam \edtext{cognitio}{\lemma{cognitio}\Bfootnote{intellectio S}} aenigmatica. Ideo dicit  \name{Apostolus}\index[persons]{}:  \edtext{\enquote{nunc autem videmus per speculum et in aenigmate, tunc autem faciem ad faciem,}}{\lemma{}\Afootnote[nosep]{
\crossref{\xlineref{http://scta.info/resource/Icor13_12}}{\worktitle{I Ad Corinthios}\index[works]{} 13:12}}}\index[sources]{x!Bible!ad Corinthios I} etc.}{\lemma{Unde \dots etc.}\Bfootnote{dicitur in praedicta diffinitione non apparentium, 
quia non oportet quod subiectum sit apparentium, 
et de hoc aliquid dicetur V}} 
\pend

\pstart
\ledsidenote{\textbf{5}}
\edlabel{http://scta.info/resource/l1-uaqriv}
\edtext{Ubi advertendum quod}{\lemma{Ubi advertendum quod}\Bfootnote{et V}} de hac materia sunt \edtext{tres}{\lemma{tres}\Bfootnote{quinque V}} \edtext{diversae viae ad imaginandum quomodo quis inducitur ad credendum articulis fidei}{\lemma{diversae \dots fidei}\Bfootnote{viae V}}: duae extremae et tertia media. \edtext{Prima est quae est tacta}{\lemma{}\Afootnote[nosep]{Guillelmus Altissiodorensis, Liber Primus, prologus (ad Claras Aquas, I:15, l. 2-3): "Sicut enim vera dilectione diligitur Deus propter seipsum super omnia, ita fide acquiescitur prime veritati super omnia propter se."}}\index[sources]{!!}: quod \edtext{intellectus}{\lemma{intellectus}\Bfootnote{non R SV}} per fidem assentit primae veritati propter se sine quacumque apparentia. \edtext{Secunda}{\lemma{Secunda}\Bfootnote{Quarta via V}} est \edtext{\name{Holcot}\index[persons]{}}{\lemma{}\Afootnote[nosep]{Cf. Holcot, Commentary, prologue}}\index[sources]{!!} \edtext{primae}{\lemma{primae}\Bfootnote{praedictae V}} extremae contrariatur, scilicet, quod assensus fidei nullo modo generatur, nisi per rationem cogentem intellectum, \edtext{non affectatum ad oppositum; et ultra credere articulis non est meritorium, quia non est in libera potestate voluntatis, quia quis ad credendum articulis necessatur  
\ledsidenote{S2-va}
quandoque}{\lemma{non \dots quandoque}\Bfootnote{ V}} Tertia, via media, est quam \edtext{determinat}{\lemma{determinat}\Bfootnote{declarat V}} \edtext{\name{de Heuta}\index[persons]{}}{\lemma{}\Afootnote[nosep]{Cf. Oyta}}\index[sources]{!!} quod 
\ledsidenote{V5-vb}
requiritur ratio probabilis; non tamen necessitans ad credendum, et ideo cum hoc requiritur imperium voluntatis.
\pend

\bigskip
\section*{\textlangle{}Argumenta pro prima via ad imaginandum fidem\textrangle{}} 
\pstart
\ledsidenote{\textbf{6}}
\edlabel{http://scta.info/resource/l1-pveain}
\edtext{Prima via est \name{Altissiodorensis}\index[persons]{} ponentis quod intellectus adiutus lumine fidei potest assensum articulorum in se causare sine aliqua apparentia extrinseca vel ratione suadente vel inductiva, immo ipsum lumen fidei et intellectus sufficiunt ad causandum assensum articulorum in nobis}{\lemma{Prima \dots nobis}\Bfootnote{\textit{om.} V}}. 
\pend

\pstart
\ledsidenote{\textbf{7}}
\edlabel{http://scta.info/resource/l1-pivqep}
Prima \edtext{ista}{\lemma{ista}\Bfootnote{\textit{om.} V}} via imaginatur conformiter ad \edtext{\edtext{\name{Augustinum}\index[persons]{}}{\lemma{\name{Augustinum}\index[persons]{}}\Bfootnote{argumentum R SV}} XIIo \worktitle{de civitate Dei}\index[works]{}}{\lemma{}\Afootnote[nosep]{Augustine, \worktitle{City of God}\index[works]{},??}}\index[sources]{!!}, qui ponit \edtext{originem}{\lemma{originem}\Bfootnote{ R SV S}} duarum civitatum, scilicet, Dei et diaboli, ex parte \edtext{voluntatis.}{\lemma{voluntatis.}\Bfootnote{et V}} Ponit \edtext{ibi \name{Augustinus}\index[persons]{}}{\lemma{}\Afootnote[nosep]{Augustine, \worktitle{City of God}\index[works]{}??}}\index[sources]{!!}  
\ledsidenote{R1-va}
quod duo amores \edtext{fecerunt}{\lemma{fecerunt}\Bfootnote{fecerunt R SV}} duas civitates: civitatem Dei fecit \edtext{fecit}{\lemma{fecit}\Bfootnote{facit R}} amor Dei usque ad \edtext{contemptum}{\lemma{contemptum}\Bfootnote{contentum V}} sui; civitatem diaboli \edtext{fecit}{\lemma{fecit}\Bfootnote{\textit{om.} R SV S}} amor sui usque ad \edtext{contemptum}{\lemma{contemptum}\Bfootnote{contentum V}} Dei. Huiusmodi ergo homines qui talem habent amorem  
\ledsidenote{SV187-va}
\edtext{erga se}{\lemma{erga se}\Bfootnote{ergo R  erga S SV}} sunt \edtext{cives civitatis}{\lemma{cives civitatis}\Bfootnote{omnes R SV S}} diaboli\edtext{}{\lemma{diaboli}\Bfootnote{et \textit{in textu} R}}. \edtext{Consimiliter ponit}{\lemma{Consimiliter ponit}\Bfootnote{conformiter posset poni V}} de intellectu respectu veritatum, quia quaedam sunt veritates commodae intellectui etiam in statu naturae lapsae. Alia autem est \edtext{veritas}{\lemma{veritas}\Bfootnote{virtus R SV}} ad quam intellectus, seclusis omnibus aliis veritatibus et apparentiis, fertur in illam veritatem, ita quod totaliter seipsum abnegat propter primam veritatem. Et sic erit cuius civitas Dei, sicut dictum est de voluntate, et talis est maxima perfectio ipsius intellectus, scilicet, omnino se negare, sicut credere istam: \enquote*{Deus est}, quae est prima.
\pend

\pstart
\ledsidenote{\textbf{8}}
\edlabel{http://scta.info/resource/l1-eqsqre}
Ex \edtext{quo}{\lemma{quo}\Bfootnote{qua V}} statim \edtext{sequitur}{\lemma{sequitur}\Bfootnote{eo \textit{add. sed del.} V}} quod remunerator est eo quod \edtext{bonus est et}{\lemma{bonus est et}\Bfootnote{\textit{om.} V}} summe bonus est et eo quod summe diligit creaturam et non summe diligeret nisi remuneraret bene appetentem, et hoc notavit  \name{Apostolus}\index[persons]{}, dicens quod \edtext{\enquote{oportet accendentem credere quia est, et quia remunerator \edtext{est}{\lemma{est}\Bfootnote{\textit{om.} V}}}}{\lemma{}\Afootnote[nosep]{
\crossref{\xlineref{http://scta.info/resource/hebr11_6}}{\worktitle{Ad Hebraeos}\index[works]{} 11:6}}}\index[sources]{x!Bible!ad Hebraeos}. 
\pend

\bigskip
\section*{\textlangle{}Corollaria primae viae\textrangle{}} 
\pstart
\ledsidenote{\textbf{9}}
\edlabel{http://scta.info/resource/l1-eqsfss}
Ex quibus sequitur primo quod ex rationibus probabilibus non generatur fides; patet quia fides et actus fidei sunt supernaturales.
\pend

\pstart
\ledsidenote{\textbf{10}}
\edlabel{http://scta.info/resource/l1-ssqsdd}
Secundo sequitur quod nec rationes evidentes sufficiunt concurrere ad \edtext{tam}{\lemma{tam}\Bfootnote{eam R SV}} \edtext{nobilis habitus et assensus}{\lemma{nobilis habitus et assensus}\Bfootnote{nobiles actus V}} productionem. Id est, rationes demonstrativae non possunt causare actum fidei, et ponendum est in claris terminis, quia \edtext{Sacra}{\lemma{Sacra}\Bfootnote{\textit{om.} V}} Scriptura elucidanda est et non obscuranda iuxta illud, \edtext{\enquote{qui elucidant me vitam aeternam habebunt}}{\lemma{}\Afootnote[nosep]{
\crossref{\xlineref{http://scta.info/resource/sir24_31}}{\worktitle{Sirach}\index[works]{} 24:31}}}\index[sources]{x!Bible!Ecclesiasticus}. Et si obiciatur de hoc, quod refert  \name{Macrobius}\index[persons]{}: quod \edtext{\enquote{quidam 
\ledsidenote{S2-vb}
vidit musas poeticas lacrimantes eo quod ea quae scribuntur \edtext{in Scriptura Sacra scribuntur}{\lemma{in Scriptura Sacra scribuntur}\Bfootnote{\textit{om.} R SV  scribuntur V}} sine ornatu earum}}{\lemma{}\Afootnote[nosep]{Non invenimus adhuc.}}\index[sources]{!!}. Respondendum \edtext{est}{\lemma{est}\Bfootnote{ R}} quod scientiae,  
\ledsidenote{V6-ra}
quae ad superbia \edtext{inducunt}{\lemma{inducunt}\Bfootnote{inductione S}}, \edtext{sicut}{\lemma{sicut}\Bfootnote{cuius V}} sunt scientiae humanitus adinventae, debent tegi verbis, ne omnibus \edtext{appareant}{\lemma{appareant}\Bfootnote{pateant V}}. Scriptura autem Sacra, quae ad humilitatem inducit, velari non \edtext{debet}{\lemma{debet}\Bfootnote{debent V}}, sed \edtext{debet}{\lemma{debet}\Bfootnote{\textit{om.} V}} declarari.
\pend

\pstart
\ledsidenote{\textbf{11}}
\edlabel{http://scta.info/resource/l1-tsqnee}
Tertio sequitur quod lux fidei est \edtext{superioris}{\lemma{superioris}\Bfootnote{superior S}} speciei et ex parte sui certior quam tota latitudo omnium rationum \edtext{naturalium}{\lemma{naturalium}\Bfootnote{materialium S}}. Patet \edtext{quia,}{\lemma{quia,}\Bfootnote{\textit{om.} V}} propter propinquitatem quam habet ad summam lucem, ipsa est certior et excellentior quam latitudo tota luminis \edtext{naturalis,}{\lemma{naturalis,}\Bfootnote{materialis S}} et sic rationes probabiles non sufficiunt \edtext{in}{\lemma{in}\Bfootnote{in \textit{add.} S}} nobis generare habitum fidei. Ex isto sequitur quod fides \edtext{excedit}{\lemma{excedit}\Bfootnote{excedit \textit{corr. ex} exceditis R}} primum principium in firmitate, cum primum principium sit infra latitudinem luminis naturalis, igitur\edtext{}{\lemma{igitur}\Bfootnote{\textit{iterum} S}}. 
\pend

\pstart
\ledsidenote{\textbf{12}}
\edlabel{http://scta.info/resource/l1-scoesi}
Sed contra, \edtext{obicitur}{\lemma{obicitur}\Bfootnote{obiceretur V S}} quia prima principia sunt nobis innata et \edtext{sunt}{\lemma{sunt}\Bfootnote{sint S}} evidentissima et fides non sic, ergo in esse claritatis\edtext{}{\lemma{claritatis}\Bfootnote{vid \textit{add. sed del.} SV}}, fides videtur 
\ledsidenote{R1-vb}
esse speciei inferioris.
\pend

\pstart
\ledsidenote{\textbf{13}}
\edlabel{http://scta.info/resource/l1-chpnpn}
Contra hoc ponitur quartum corollarium, \edtext{iuxta immediate dicta}{\lemma{iuxta immediate dicta}\Bfootnote{\textit{om.} V}}: quod quantum ad hoc, fides excedit primum principium complexum non solum in luce, sed etiam in firmitate. Illud non probatur nunc.
\pend

\pstart
\ledsidenote{\textbf{14}}
\edlabel{http://scta.info/resource/l1-qsqfrf}
Quinto sequitur quod facilius esset dissentire primo principio quam fidei. Istud fundatur primo in supremo modo se habendi\edtext{}{\lemma{habendi}\Bfootnote{illius \textit{in textu} V}} intellectus; 
\ledsidenote{SV187-vb}
secunda radix est quia intellectus, adhaerens primae veritati, abnegat seipsum et omnem rationem et sic est \edtext{immobilis}{\lemma{immobilis}\Bfootnote{immutabilis V}} quantum ad hoc. Ideo, intellectus sic fide illustratus, secundum \name{Altissiodorensis}\index[persons]{},  potest dicere Samaritanae, id est, rationi humanae, \edtext{\enquote{iam \edtext{non}{\lemma{non}\Bfootnote{nota R}} propter te credimus, sed quod ipsum\edtext{}{\lemma{ipsum}\Bfootnote{\textit{iterum} V}} \edtext{vidimus}{\lemma{vidimus}\Bfootnote{vidimus \textit{corr. ex} vendimus V}} et audivimus}}{\lemma{}\Afootnote[nosep]{
\crossref{\xlineref{http://scta.info/resource/io4_42}}{\worktitle{Ioannis}\index[works]{} 4:42}}}\index[sources]{x!Bible!Ioannes}. Sed de primo principio possunt adduci rationes sophisticae per quas ligabitur, nec ibi abnegat \edtext{omnem}{\lemma{omnem}\Bfootnote{creaturam \textit{add. sed del.} SV}} rationem quemadmodum facit \edtext{respectu}{\lemma{respectu}\Bfootnote{ratione V}} fidei.
\pend

\pstart
\ledsidenote{\textbf{15}}
\edlabel{http://scta.info/resource/l1-eqsosp}
Ex quo sequitur \edtext{sexto}{\lemma{sexto}\Bfootnote{\textit{om.} V  quinto S}}: quod adhaesio fidei tollit omne  
\ledsidenote{S3-ra}
sophisticum periculum.
\pend

\pstart
\ledsidenote{\textbf{16}}
\edlabel{http://scta.info/resource/l1-ssqper}
Septimo sequitur\edtext{}{\lemma{sequitur}\Bfootnote{videtur \textit{in textu} SV}} quod rationes tam pro se quam contra se \edtext{contemnit}{\lemma{contemnit}\Bfootnote{\textit{om.} S}}, \edtext{\edtext{ne}{\lemma{ne}\Bfootnote{quantum est ad hoc ut V  pro se ne SV}} per \edtext{eas}{\lemma{eas}\Bfootnote{ea R SV}} \edtext{quae}{\lemma{quae}\Bfootnote{\textit{om.} V}} credat -- quod alias variarent habitum creditum \edtext{contra se}{\lemma{contra se}\Bfootnote{\textit{om.} V}},}{\lemma{ne \dots,}\Bfootnote{\textit{om.} S}} \edtext{ne per eas \edtext{reducatur}{\lemma{reducatur}\Bfootnote{seducatur S}}}{\lemma{ne per eas \edtext{reducatur}{\lemma{reducatur}\Bfootnote{seducatur S}}}\Bfootnote{ V}}. 
\pend

\pstart
\ledsidenote{\textbf{17}}
\edlabel{http://scta.info/resource/l1-osqrnr}
Octavo sequitur quod fides quantum est de se omnem rationem naturalem repudiat.
\pend

\bigskip
\section*{\textlangle{}Obiectio ad primam viam ad imaginandum fidem et responsio Petri Plaoul\textrangle{}} 
\pstart
\ledsidenote{\textbf{18}}
\edlabel{http://scta.info/resource/l1-ctindr}
Contra tamen istud obicitur, quia tunc doctores et sancti peccarent, quia, pro fide \edtext{defensione}{\lemma{defensione}\Bfootnote{deffectione S}}, \edtext{multas fecerunt}{\lemma{multas fecerunt}\Bfootnote{multum faciunt V}} rationes, sicut \edtext{patet}{\lemma{patet}\Bfootnote{\textit{om.} R}} in 
\ledsidenote{V6-rb}
\edtext{isto libro \worktitle{Sententiarum}\index[works]{}}{\lemma{}\Afootnote[nosep]{Lombard, \worktitle{Sentences}\index[works]{} I}}\index[sources]{!!} et aliis libris doctorum. \edtext{Unde dicit  \name{Augustinus}\index[persons]{}  quod \edtext{\enquote{scientia theologiae et Sacrae Scripturae est saluberrima, quae fides gignitur, nutritur, defenditur, roboratur}}{\lemma{}\Afootnote[nosep]{
\crossref{\xlineref{http://scta.info/resource/adt-l14-d1e1535@92-108}}{\name{Augustinus}\index[persons]{} \worktitle{De Trinitate}\index[works]{} XIV, c. 1}}}\index[sources]{!!}.}{\lemma{Unde \dots.}\Bfootnote{\textit{om.} V}} 
\pend

\pstart
\ledsidenote{\textbf{19}}
\edlabel{http://scta.info/resource/l1-rahrah}
\edtext{Respondeo}{\lemma{Respondeo}\Bfootnote{respondetur V}} \edtext{ad hoc}{\lemma{ad hoc}\Bfootnote{\textit{om.} V}}: 
\pend

\pstart
\ledsidenote{\textbf{20}}
\edlabel{http://scta.info/resource/l1-qrpcdi}
\textlangle{}Primo\textrangle{} quod rationes probabiles \edtext{multipliciter}{\lemma{multipliciter}\Bfootnote{multum V}} \edtext{valent}{\lemma{valent}\Bfootnote{unde R SV}}, non tamen ad fidei generationem, quia rationes tales sunt propter \edtext{alias rationes}{\lemma{alias rationes}\Bfootnote{alia V}}, quae possent fieri ex \edtext{\textlangle{}insulto\textrangle{}}{\lemma{\textlangle{}insulto\textrangle{}}\Bfootnote{insultu R V S SV}} diaboli. \edtext{Ideo valent contra diaboli insultum.}{\lemma{Ideo valent contra diaboli insultum.}\Bfootnote{\textit{om.} V}} 
\pend

\pstart
\ledsidenote{\textbf{21}}
\edlabel{http://scta.info/resource/l1-senpht}
Secundo, ex naturali dispositione hominis post lapsum, vires \edtext{eius}{\lemma{eius}\Bfootnote{eum S}} sensitivae sunt variae. Ideo veniunt multae cogitationes et \edtext{multa}{\lemma{multa}\Bfootnote{\textit{om.} R SV S}} phantasmata contra fidem; rationes autem probabiles hoc tollunt.
\pend

\pstart
\ledsidenote{\textbf{22}}
\edlabel{http://scta.info/resource/l1-tvaaec}
\edtext{Tertio valent}{\lemma{Tertio valent}\Bfootnote{primo V}} ad reflectendum contra haereticos et gentiles rationes suas et \edtext{et}{\lemma{et}\Bfootnote{in S}} \edtext{removendo}{\lemma{removendo}\Bfootnote{rememorando V}} \edtext{eorum}{\lemma{eorum}\Bfootnote{earum V}} fallacias et ad eos \edtext{convincendum}{\lemma{convincendum}\Bfootnote{convertendum V}}. 
\pend

\pstart
\ledsidenote{\textbf{23}}
\edlabel{http://scta.info/resource/l1-qpehio}
Quarto possunt esse tales rationes quod inclinabunt ad \edtext{assentiendum}{\lemma{assentiendum}\Bfootnote{assentiendum \textit{corr. ex} ssentiendum S}} firmiter \edtext{articulis}{\lemma{articulis}\Bfootnote{articulos S}} fidei, \edtext{quia ex illis potest generari habitus qui inclinabit ad assentiendum articulis. Tamen talis habens habitum talem,}{\lemma{quia \dots talem,}\Bfootnote{\textit{om.} V}} \edtext{non habens tamen infusum habitum fidei,}{\lemma{non habens tamen infusum habitum fidei,}\Bfootnote{non tamen haberet fidem et sic tales V}} non \edtext{esset fidelis}{\lemma{esset fidelis}\Bfootnote{essent fideles V}}, nec etiam \edtext{haberet}{\lemma{haberet}\Bfootnote{haberent V}} habitum infidelitatis oppositum.
\pend

\pstart
\ledsidenote{\textbf{24}}
\edlabel{http://scta.info/resource/l1-qqsdai}
Quinto quia sunt manuductivae ad fidem aliqualiter \edtext{et}{\lemma{et}\Bfootnote{\textit{om.} V}} dispositivae ad ipsam.
\pend

\pstart
\ledsidenote{\textbf{25}}
\edlabel{http://scta.info/resource/l1-spciaf}
\edtext{Sexto}{\lemma{Sexto}\Bfootnote{quinto S}} pie credendo per \edtext{\edtext{per}{\lemma{per}\Bfootnote{quod S}} tales rationes,}{\lemma{\edtext{per}{\lemma{per}\Bfootnote{quod S}} tales rationes,}\Bfootnote{\textit{om.} V}} verisimile est quod Deus suppleat residuum ex quo homo facit diligentiam in acquirendo fidem.
\pend

\pstart
\ledsidenote{\textbf{26}}
\edlabel{http://scta.info/resource/l1-ssqfes}

\ledsidenote{R2-ra}
Septimo staret quod quis crederet primo propter rationem, postea\edtext{}{\lemma{postea}\Bfootnote{vero \textit{in textu} V}} \edtext{propter}{\lemma{propter}\Bfootnote{per V}} fidem, sicut stat quod aliquis incipiat aliquid habendo respectum ad malum finem et tandem continuando \edtext{haberet}{\lemma{haberet}\Bfootnote{habebit V}} respectum ad bonum finem. Sicut patet de \edtext{eundo}{\lemma{eundo}\Bfootnote{eunte V}} ad ecclesiam pro lucrando distributiones, 
\ledsidenote{S3-rb}
qui postea \edtext{esset ibi propter Deum, ut si aliquis canonicus vadat ad ecclesiam ea  
\ledsidenote{SV188-ra}
sola intentione vel affectione ut distributiones suas recipiat; et licet ista affectio non sit meritoria, tamen virtute illius inductionis ad eundum ad ecclesiam, dum ibit vel dum erit \edtext{ibi}{\lemma{ibi}\Bfootnote{\textit{om.} R}} in ecclesia, potest habere \edtext{rectam}{\lemma{rectam}\Bfootnote{causam R}} intentionem, ut ibi sit et maneat, non propter distributiones (licet propter eas venerit), sed ut divinum servitium cum devotione audiat vel ibi oret vel matutinas cum aliis cum devotione decantet. Et ista possunt esse meritoria, licet principaliter, dum venerit, hoc \textlangle{}non\textrangle{} intenderet. Et sic rationes sunt manuductivae ad credendum et consequenter ad errorem diminuendum}{\lemma{esset \dots diminuendum}\Bfootnote{cantat et est in devotione propter Deum V}}, sicut est de pagano qui converteretur ad fidem propter\edtext{}{\lemma{propter}\Bfootnote{aliquod \textit{in textu} V}} bonum temporale et postea Deo crederet propter \edtext{seipsum}{\lemma{seipsum}\Bfootnote{ipsum V}}. \edtext{Sic etiam rationes probabiles sunt manuductivae ad credendum alicui articulo, postea poterit talis percipere quod non propter rationes vel apparentias est credendum, sed mere libere non \edtext{intendendo}{\lemma{intendendo}\Bfootnote{intendo S}} rationi humanae, et sic credendo mereretur, \edtext{alias}{\lemma{alias}\Bfootnote{vel S}} non. Iuxta idem, \edtext{idem}{\lemma{idem}\Bfootnote{illud S}} fides non habet meritum \edtext{cui}{\lemma{cui}\Bfootnote{ubi R}} humana ratio praebet experimentum; sic forsan postea fideliter credet, non tamen propter rationes, ut li \enquote*{propter} dicit circumstantiam causae efficientis; bene tamen si li \enquote*{propter} dicit circumstantiam occasionis.}{\lemma{Sic \dots occasionis.}\Bfootnote{\textit{om.} V}} 
\pend

\pstart
\ledsidenote{\textbf{27}}
\edlabel{http://scta.info/resource/l1-ivafes}
\edtext{Item, \textlangle{}octavo\textrangle{}, valent ad infirmitatem humani ingenii sanandam quod quid ingenium propter peccatum culpae originalis depressum est et volneratum; et roborandum relevandum est per luminem supernaturale}{\lemma{Item, \dots supernaturale}\Bfootnote{\textit{om.} V}}. 
\pend

\pstart
\ledsidenote{\textbf{28}}
\edlabel{http://scta.info/resource/l1-pirfes}
Propter \edtext{istas}{\lemma{istas}\Bfootnote{praedictas V}} rationes, non in merito, \edtext{praecipit}{\lemma{praecipit}\Bfootnote{procedit V}}  \name{Apostolus Petrus}\index[persons]{}  specialiter \edtext{praelatis}{\lemma{praelatis}\Bfootnote{clericis R}} ut: \edtext{\enquote{parati sint omni poscenti reddere rationem de ea quae in eis est fide et spe}}{\lemma{}\Afootnote[nosep]{
\crossref{\xlineref{http://scta.info/resource/Ipetr3_15}}{I Petrus 3:15}}}\index[sources]{x!Bible!Petri I}. 
\pend

\bigskip
\section*{\textlangle{}Aliqua conclusiones\textrangle{}} 
\bigskip
\section*{\textlangle{}Prima conclusio\textrangle{}} 
\pstart
\ledsidenote{\textbf{29}}
\edlabel{http://scta.info/resource/l1-equeri}
Ex quibus ultra sequitur quod: fides est proprie sanitas intellectus, sicut peccatum \edtext{fuit}{\lemma{fuit}\Bfootnote{est V}} \edtext{morbus ipsius.}{\lemma{morbus ipsius.}\Bfootnote{morbum ipsius SV  morbus ipsum S  morbum ipsius V}}. Per peccatum enim fuit non solum voluntas \edtext{volnerata}{\lemma{volnerata}\Bfootnote{viciata V}}, sed etiam intellectus; per fidem autem sanatur intellectus, sicut voluntas per caritatem. Ideo est dura  
\ledsidenote{V6-va}
a principio. Unde et apostolis dura fuit, dicebant enim: \edtext{\enquote{Durus est hic sermo et \edtext{quis}{\lemma{quis}\Bfootnote{qui V}} potest eum audire}}{\lemma{}\Afootnote[nosep]{
\crossref{\xlineref{http://scta.info/resource/io6_60}}{Iohannes 6:60}}}\index[sources]{x!Bible!Ioannes}? Et \edtext{ratio}{\lemma{ratio}\Bfootnote{ideo V}} est  
\ledsidenote{S3-va}
quia dividit animam a corporibus elevando eam usque ad \edtext{contemptum}{\lemma{contemptum}\Bfootnote{contentum V}} sui. Unde  \name{Apostolus}\index[persons]{}: \edtext{\enquote{Vivus et efficax est sermo Dei et penetrabilior omni gladio \edtext{ancipiti}{\lemma{ancipiti}\Bfootnote{ancipiter R S SV}}}}{\lemma{}\Afootnote[nosep]{
\crossref{\xlineref{http://scta.info/resource/hebr4_12}}{Ad Hebraeos 4:12}}}\index[sources]{x!Bible!ad Hebraeos}. \edtext{Unde sicut 
\ledsidenote{R2-rb}
potio amara videtur dura et amara in principio aegrotanti, in fine tamen est sana, sic fides; nam in principio durum videtur credere articulis \edtext{eam intuentibus, in fine}{\lemma{eam intuentibus, in fine}\Bfootnote{causa inevidentibus S}} tamen fides fortificat et roborat intellectum}{\lemma{Unde \dots intellectum}\Bfootnote{\textit{om.} V}}. 
\pend

\bigskip
\section*{\textlangle{}Secunda conclusio\textrangle{}} 
\pstart
\ledsidenote{\textbf{30}}
\edlabel{http://scta.info/resource/l1-acfefv}
Alia conclusio: \edtext{fides}{\lemma{fides}\Bfootnote{\textit{om.} V}} non solum est meritoria, \edtext{immo}{\lemma{immo}\Bfootnote{sed V}} formaliter est meritum. Quod sit 
\ledsidenote{SV188-rb}
meritoria ab omnibus conceditur. Quod sit \edtext{formaliter}{\lemma{formaliter}\Bfootnote{\textit{om.} V}} meritum patet, \edtext{licet credere sit ex parte actus intellectus ut declarabo. Patet ergo quod fides sit meritum}{\lemma{licet \dots meritum}\Bfootnote{\textit{om.} V}}  \edtext{per \name{Apostolum}\index[persons]{}}{\lemma{per \name{Apostolum}\index[persons]{}}\Bfootnote{\textit{om.} R SV S}}: \edtext{\enquote{\name{Abraham}\index[persons]{} credidit et reputatum est \edtext{sibi}{\lemma{sibi}\Bfootnote{\textit{om.} V}} ad iustitiam}}{\lemma{}\Afootnote[nosep]{
\crossref{\xlineref{http://scta.info/resource/rom4_3}}{Ad Romanos 4:3}}}\index[sources]{x!Bible!ad Romanos}, et  \worktitle{Habacuc}\index[works]{}: \edtext{\enquote{iustus ex fide vivit}}{\lemma{}\Afootnote[nosep]{
\crossref{\xlineref{http://scta.info/resource/hab2_4}}{\worktitle{Prophetia Habacuc}\index[works]{} 2:4}}}\index[sources]{x!Bible!Habacuc}. 
\pend

\pstart
\ledsidenote{\textbf{31}}
\edlabel{http://scta.info/resource/l1-pqeqra}
Pro quo \edtext{est}{\lemma{est}\Bfootnote{\textit{om.} R S SV}} notandum quod cum eadem res sit intellectus et voluntas omnino, tamen alius est actus per quem anima dicitur intellectus, \edtext{scilicet intellectio,}{\lemma{scilicet intellectio,}\Bfootnote{\textit{om.} R SV S}} et alius per\edtext{}{\lemma{per}\Bfootnote{quae \textit{in textu} S}} quem dicitur voluntas, scilicet, volitio. Ideo tres possunt distingui actus in anima: quiddam sunt cognitivi solum, ut intellectiones; quiddam appetivi solum, ut volitiones; quiddam vero mixti ex praedictis et in talibus consistit fides, caritas, et spes et eorum actus. Unde sicut actus caritatis est actus voluntatis, \edtext{et}{\lemma{et}\Bfootnote{\textit{om.} V}} tamen est \edtext{quodammodo}{\lemma{quodammodo}\Bfootnote{quaedam V}} perceptio obiecti et sic pertinet ad intellectum, principalius tamen ad voluntatem, \edtext{sic e converso, principalius fides pertinet et est actus intellectus, etiam pertinet ad voluntatem}{\lemma{sic \dots voluntatem}\Bfootnote{\textit{om.} V}}. Et ideo sicut creatura obligatur ad actus caritatis, ita ad actus fidei. Et sicut per diligere meretur debito modo \edtext{diligendo}{\lemma{diligendo}\Bfootnote{\textit{om.} V}}, sic \edtext{etiam}{\lemma{etiam}\Bfootnote{\textit{om.} R SV S}} per credere debito modo meretur, et hoc concordat dictis  \name{Augustini}\index[persons]{}  dicentis: \edtext{\enquote{Cetera potest homo \edtext{nolens, credere non nisi volens.}{\lemma{nolens, credere non nisi volens.}\Bfootnote{volens credere aut non ut volens V}}}}{\lemma{}\Afootnote[nosep]{\name{Augustinus}\index[persons]{}, \worktitle{Tractatus XXVI in Ioannem}\index[works]{}, 2}}\index[sources]{!!}. Item, sicut unum crescit, sic et reliquum nec videtur plus repugnare \edtext{augmentatio}{\lemma{augmentatio}\Bfootnote{augmentari V}} intellectiva respectu unius quam respectu alterius.
\pend

\bigskip
\section*{\textlangle{}Obiectio et responsio\textrangle{}} 
\pstart
\ledsidenote{\textbf{32}}
\edlabel{http://scta.info/resource/l1-scosec}
Sed contra obiceretur bene apparenter quia in daemonibus est fides, \edtext{quia}{\lemma{quia}\Bfootnote{qui V}}\edtext{}{\lemma{quia}\Bfootnote{quod \textit{add. sed del.} S}} \edtext{\enquote{credunt et contremiscunt}}{\lemma{}\Afootnote[nosep]{
\crossref{\xlineref{http://scta.info/resource/iac2_19}}{Iacobus 2:19}}}\index[sources]{x!Bible!Iacobi} et tamen non habent caritatem, ergo etc. Item  \edtext{\name{Iacobus}\index[persons]{}}{\lemma{\name{Iacobus}\index[persons]{}}\Bfootnote{Iacobum SV}}  dicit quod: \edtext{\enquote{fides 
\ledsidenote{S3-vb}
sine operibus mortua est}}{\lemma{}\Afootnote[nosep]{
\crossref{\xlineref{http://scta.info/resource/iac2_17}}{Iacobus 2:17}}}\index[sources]{x!Bible!Iacobi}, ergo non vivificat ex se. Item  \name{Apostolus}\index[persons]{}  dicit: \edtext{\enquote{si habuero \edtext{omnem}{\lemma{omnem}\Bfootnote{\textit{om.} V}} fidem, ita ut montes transferam, caritatem autem non \edtext{habuero}{\lemma{habuero}\Bfootnote{habens V}}, nihil sum}}{\lemma{}\Afootnote[nosep]{
\crossref{\xlineref{http://scta.info/resource/Icor13_2}}{\worktitle{1 Ad Corinthios}\index[works]{} 13:2}}}\index[sources]{x!Bible!ad Corinthios I}, ergo contradictam.
\pend

\pstart
\ledsidenote{\textbf{33}}
\edlabel{http://scta.info/resource/l1-airqip}
Ad ista respondetur sine temeraria assertione. Primo imaginando duplicem fidem; unam acquisitam 
\ledsidenote{V6-vb}
per rationes humanas et forte \edtext{derelictam}{\lemma{derelictam}\Bfootnote{derelicta R SV S}} ex actibus credendi. Et ista staret cum peccato, sicut in simili ponimus de caritate acquisita; cum \edtext{enim}{\lemma{enim}\Bfootnote{autem V}} quis peccat\edtext{}{\lemma{peccat}\Bfootnote{et \textit{in textu} V}} prius \edtext{autem}{\lemma{autem}\Bfootnote{\textit{om.} V}} \edtext{dilexerit}{\lemma{dilexerit}\Bfootnote{dilexit S}} Deum caritative, se sentit pronum adhuc ad diligendum Deum et ista \edtext{pronitas}{\lemma{pronitas}\Bfootnote{caritas V}} non provenit ex caritate infusa, quia non stat cum peccato, sed ex caritate acquisita. Et talis fides bene est in daemonibus, et de ista  
\ledsidenote{R2-va}
possunt intelligi auctoritates praedictae. Alia \edtext{vero}{\lemma{vero}\Bfootnote{\textit{om.} V}} est fides infusa, de qua intelligebantur praedicta.
\pend

\pstart
\ledsidenote{\textbf{34}}
\edlabel{http://scta.info/resource/l1-ampeno}
Alio modo potest \edtext{responderi}{\lemma{responderi}\Bfootnote{tripliciter V}} imaginando quod fides multiplicem habet habitudinem ad suum subiectum; \edtext{aliam inhaerendi,}{\lemma{aliam inhaerendi,}\Bfootnote{ V}} aliam, scilicet, \edtext{afficiendi}{\lemma{afficiendi}\Bfootnote{assentiendi V}}, et aliam repraesentandi. \edtext{Et}{\lemma{Et}\Bfootnote{\textit{om.} S}} \edtext{tunc\edtext{}{\lemma{tunc}\Bfootnote{indignitatis creaturae auffert duas secundas habitudines et non primam 
et sic illa et sic illa fides mortua est \textit{add. sed del.} SV}} diceretur}{\lemma{tunc \dots diceretur}\Bfootnote{sic dicendum est V}} quod \edtext{possibile}{\lemma{possibile}\Bfootnote{impossibile V}} est absolute stare primam habitudinem aliis duabus \edtext{subtractis,}{\lemma{subtractis,}\Bfootnote{abstractis V}} quia diceretur quod Deus, ratione indignitatis creaturae, \edtext{aufert}{\lemma{aufert}\Bfootnote{auferret V}} duas secundas habitudines et non prima. Et sic illa fides mortua est, quia non exit in operationes \edtext{suas}{\lemma{suas}\Bfootnote{sibi V}} proprias; 
\ledsidenote{SV188-rb}
res autem dicitur mortua quando in proprias operationes exire non potest, ita quod Deus non vellet concurrere ad eius nobiles operationes.
\pend

\bigskip
\section*{\textlangle{}Duas difficultates de merito fidei\textrangle{}} 
\pstart
\ledsidenote{\textbf{35}}
\edlabel{http://scta.info/resource/l1-strefm}
Sed tunc restat difficultas, quia obiectum intellectus est verum vel apparens verum, igitur non fertur, nisi in tale. Secundo, intellectus realiter non est liber, vel saltem eius operatio non est immediate libera, igitur non est formaliter meritum.
\pend

\pstart
\ledsidenote{\textbf{36}}
\edlabel{http://scta.info/resource/l1-qailaa}
Quantum ad istud secundum, de liberate, non esset impossibile, quia agentia naturalia haberent instinctum naturalem libertatis excepta cognitione. \edtext{Unde}{\lemma{Unde}\Bfootnote{una V}} de igne respectu stuppae potest Deus instituere quod omnibus extrinsecis \edtext{omnino}{\lemma{omnino}\Bfootnote{\textit{om.} V}} se habentibus eodem modo \edtext{posset}{\lemma{posset}\Bfootnote{potest S}} comburere \edtext{vel}{\lemma{vel}\Bfootnote{et V}} non comburere. Alio modo diceretur quod actus, qui simul est cognitio et quaedam affectio,  
\ledsidenote{S4-ra}
potest produci libere ab anima.
\pend

\pstart
\ledsidenote{\textbf{37}}
\edlabel{http://scta.info/resource/l1-apaeis}
Ad primum autem argumentum quod tangit de apparentia obiecti respectu intellectus, in quo est difficultas, dicendum \edtext{est}{\lemma{est}\Bfootnote{\textit{om.} S}} quod habitus fidei nedum elevat potentiam intellectivam  
\ledsidenote{V7-ra}
ad credendum, sed multipliciter fortificat ipsam respectu quarumcumque veritatum, quia \edtext{cum potentia}{\lemma{cum potentia}\Bfootnote{compositio S}} habet dispositionem \edtext{convenientem}{\lemma{convenientem}\Bfootnote{convenientissimam V}}, redditur fortior ad suas operationes exercendas, sicut videmus in elementis et \edtext{in aliis entibus}{\lemma{in aliis entibus}\Bfootnote{aliis R SV S}} naturalibus. Modo, fides est convenientissima\edtext{}{\lemma{convenientissima}\Bfootnote{ipsi \textit{in textu} V}} intellectui et tollit dispositionem \edtext{contrarium}{\lemma{contrarium}\Bfootnote{\textit{om.} S  gradus R SV}}. \edtext{Ideo}{\lemma{Ideo}\Bfootnote{Et ideo V}} per fidem, intellectus est vigorosior quam sine fide, et in isto facit fides rationes de se tantum probabiles \edtext{esse quasi}{\lemma{esse quasi}\Bfootnote{quia R SV  quod S}} demonstrativas, quia fides est intellectus sanativa.
\pend

\bigskip
\section*{\textlangle{}Aliqua conclusiones finales\textrangle{}} 
\pstart
\ledsidenote{\textbf{38}}
\edlabel{http://scta.info/resource/l1-itfeii}
Ideo \edtext{theologi}{\lemma{theologi}\Bfootnote{theologici R SV S}} fideles maxime debent pollere scientiis humanitus adinventis. Et si non polleant, \edtext{erit}{\lemma{erit}\Bfootnote{est V}} eorum defectus, quia per fidem eorum \edtext{intellectus}{\lemma{intellectus}\Bfootnote{\textit{om.} R SV S}} illustratur.
\pend

\pstart
\ledsidenote{\textbf{39}}
\edlabel{http://scta.info/resource/l1-eehnhf}
Et ex hoc redarguendus est \edtext{\name{Averroes}\index[persons]{}}{\lemma{\name{Averroes}\index[persons]{}}\Bfootnote{Averroes \textit{corr. ex} Adverroes SV}}, qui multum erroneae dicit quod \edtext{leges}{\lemma{leges}\Bfootnote{legent SV}} praebent impedimentum doctrinae, et hoc \edtext{ponit}{\lemma{ponit}\Bfootnote{videtur ponere V}} in \edtext{prologo III \worktitle{Physicorum}\index[works]{}}{\lemma{}\Afootnote[nosep]{\name{Averroes}\index[persons]{}, \worktitle{Commentarius in libris Physicorum}\index[works]{} III, prol.}}\index[sources]{!!}. Hoc autem erroneum est quoad legem Christianam. \edtext{Unde et de hoc est unus}{\lemma{Unde et de hoc est unus}\Bfootnote{unus et de hoc est R SV S}} articulus Parisiensis dicens:\edtext{}{\lemma{dicens}\Bfootnote{quod \textit{in textu} V}} \edtext{\enquote{dicere quod lex Christi impediat addiscere: \edtext{error}{\lemma{error}\Bfootnote{errori S}}.}}{\lemma{}\Afootnote[nosep]{
\crossref{\xlineref{http://scta.info/resource/z88d99-avn8u2-d1e1536}}{Ariticulus Parisiensis, 175}}}\index[sources]{Stephen Tempier!Articuli Parisius condemnati!Paragraph z88d99-avn8u2-d1e1536} De lege autem \name{Mohommad}\index[persons]{} quam aliquando 
\ledsidenote{R2-vb}
tenuit \name{Averroes}\index[persons]{}, verum est propter corruptiones in\edtext{}{\lemma{in}\Bfootnote{muu \textit{add. sed del.} R}} moribus, quae docentur in \edtext{illa}{\lemma{illa}\Bfootnote{secunda S}} lege. De lege autem \name{Iudaeorum}\index[persons]{} similiter, quia tunc deserti erant a Deo, sicut \edtext{nunc}{\lemma{nunc}\Bfootnote{\textit{om.} V}}. \edtext{Et}{\lemma{Et}\Bfootnote{ S}} licet \name{Averroes}\index[persons]{} \edtext{protestatus fuerit aliquando se esse}{\lemma{protestatus fuerit aliquando se esse}\Bfootnote{aliquando protestatus sic fuisse V}} Christianum verisimile, est tamen \edtext{eum}{\lemma{eum}\Bfootnote{\textit{om.} R SV S}} numquam habuisse fidem.
\pend

\pstart
\ledsidenote{\textbf{40}}
\edlabel{http://scta.info/resource/l1-qeeppn}
Quantum\edtext{}{\lemma{quantum}\Bfootnote{autem \textit{in textu} V}} est ex parte apparentiarum, sunt duo modi. Aliqua enim est apparentia, quae provenit ex parte obiecti et modi \edtext{repraesentandi}{\lemma{repraesentandi}\Bfootnote{necessitandi V}} ipsius; et ista est \edtext{communiter}{\lemma{communiter}\Bfootnote{quare R}} in lumine naturali, sic enim principiis assentimus\edtext{}{\lemma{assentimus}\Bfootnote{eum \textit{in textu} V}} propter evidentiam quam\edtext{}{\lemma{quam}\Bfootnote{evidentiam \textit{in textu} V}} faciunt ex suo modo enuntiandi. Ideo  \name{Aristoteles}\index[persons]{} \edtext{in \worktitle{Praedicamentis}\index[works]{} dicit}{\lemma{in \worktitle{Praedicamentis}\index[works]{} dicit}\Bfootnote{\textit{om.} R SV S}}  quod: \edtext{\enquote{ab eo quod res est \edtext{vel non est, oratio vera vel falsa dicitur}{\lemma{vel non est, oratio vera vel falsa dicitur}\Bfootnote{etc R SV S}}. }}{\lemma{}\Afootnote[nosep]{\name{Aristotles}\index[persons]{}, \worktitle{Praedicamenta}\index[works]{}, XII, 14b18-22}}\index[sources]{!!} Alia est apparentia, quae non provenit ab obiecto 
\ledsidenote{V7-rb}
sed prior est. Et secundum hoc dico quod divina evidentia est prior quam quodcumque obiectum  
\ledsidenote{S4-ra}
creatum; et ad illum modum refertur \edtext{evidentia}{\lemma{evidentia}\Bfootnote{apparentia V}} fidei, quia fides inter omnes habitus maxime imitatur divinam cognitionem \edtext{intantum}{\lemma{intantum}\Bfootnote{tantum V}} \edtext{quod sibi non potest subesse falsum}{\lemma{quod sibi non potest subesse falsum}\Bfootnote{sibi quod non poterit falsum V}}. Et \edtext{sicut}{\lemma{sicut}\Bfootnote{sic R SV}} cognitio divina se habet respectu futurorum contingentium, \edtext{sic et}{\lemma{sic et}\Bfootnote{ R SV S}} fides. \edtext{Nam}{\lemma{Nam}\Bfootnote{\textit{om.} V}} de tali futuro contingenti, si erit, \edtext{inclinavit}{\lemma{inclinavit}\Bfootnote{inclinat V}} ad \edtext{assensum illius}{\lemma{assensum illius}\Bfootnote{assentiendum sibi V}}: si non erit, non inclinavit. Et ex hoc posset oriri pulchra consideratio: utrum,  
\ledsidenote{SV188-va}
\edtext{scilicet,}{\lemma{scilicet,}\Bfootnote{ V}} fidei possit subesse falsum; et iterum, utrum Deus aliquam creaturam \edtext{possit}{\lemma{possit}\Bfootnote{posset R \textit{om.} S SV}} producere \edtext{per essentiam}{\lemma{per essentiam}\Bfootnote{pro essentia R SV S}} quae haberet consimilem conditionem sicut fides \edtext{habet}{\lemma{habet}\Bfootnote{\textit{om.} V}}. Et \edtext{tunc}{\lemma{tunc}\Bfootnote{sic V}} de illa \edtext{quaereretur}{\lemma{quaereretur}\Bfootnote{quaerit V}}, utrum \edtext{illa}{\lemma{illa}\Bfootnote{\textit{om.} S V}} posset errare vel \edtext{utrum}{\lemma{utrum}\Bfootnote{an V}} necessario esset recta.\edtext{}{\lemma{recta}\Bfootnote{Et de istis non plus pro nunc \textit{in textu} V}} 
\pend

\endnumbering


\bigskip
http://scta.info/resoure/l1-cpspfs/critical/transcription
\bigskip
\addcontentsline{toc}{chapter}{http://scta.info/resoure/l1-cpspfs/critical/transcription}

\bigskip
Editor's note: 
\bigskip

\beginnumbering

\pstart
\ledsidenote{\textbf{1}}
\edlabel{http://scta.info/resource/l1-cpspfs}
Circa prologum \worktitle{Sententiarum}\index[works]{}\edtext{}{\lemma{Sententiarum}\Bfootnote{Magistri Petri Plaoul \textit{in textu} R SV}} in quo \name{Magister}\index[persons]{} dicit  quod intentionis suae est \edtext{\enquote{munire Davidicam turrim vel potius munitam ostendere \edtext{clypeis}{\lemma{clypeis}\Bfootnote{\textit{om.} V}}}}{\lemma{}\Afootnote[nosep]{
\crossref{\xlineref{http://scta.info/resource/pll1prol-qifacc@11-18}}{ \name{Lombard}\index[persons]{}, \worktitle{Sentences}\index[works]{}, I, prologus., (I:1)}}}\index[sources]{!!} etc., quaero istam quaestionem: utrum \edtext{in}{\lemma{in}\Bfootnote{\textit{om.} R SV S}} causa iudiciali fidei \edtext{contra traditionem}{\lemma{contra traditionem}\Bfootnote{contradictionem R SV}} pure humanitus adinventam iudex idoneus ferret pro fide sententiam. 
\pend

\endnumbering




\printindex[sources]
\end{document}
